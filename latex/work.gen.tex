\documentclass[14pt,a4paper,russian]{scrartcl}
\usepackage[utf8]{inputenc} % кодировка
\usepackage[T2A]{fontenc}
\usepackage[russian,english]{babel}
\usepackage{mathtext}
\usepackage{indentfirst}    % красная строка для первого параграфа
\usepackage{misccorr}   % настройки для российской полиграфии
\usepackage{graphicx}
\usepackage{textcomp}
\usepackage{caption}
% \usepackage{times}
% \usepackage{amsmath}    % математическая нотация
% \renewcommand{\rmdefault}{ftm}
\usepackage{lastpage}

\usepackage{setspace} 
\usepackage{fancybox,fancyhdr}
% \usepackage[utf8]{inputenc}
% \usepackage[russian,english]{babel}
% \usepackage[T2A]{fontenc}
% \usepackage{indentfirst}w
\usepackage{amsmath,amssymb}
\usepackage[includefoot, heightrounded]{geometry}   % поля
\geometry{left=30mm}
\geometry{right=20mm}
\geometry{top=20mm}
\geometry{bottom=10mm}
\begin{document}
% \pagestyle{fancy}
\fancyhead[R]{Кочнов Андрей, ИУ1-62}
% \renewcommand{\onlyinsubfile}[1]{}
% \renewcommand{\notinsubfile}[1]{#1}
\captionsetup[table]{name=Таблица}

\begin{table}[h]
    \begin{center}
        \begin{tabular}{p{0.6\linewidth}p{0.4\linewidth}}
            Заготовка РПЗ по ОКП&Кочнов Андрей, ИУ1-62\\
            \hline
        \end{tabular}
    \end{center}
\end{table}
\section*{Техническое задание}
\subsubsection*{Тема: привод следящей системы (задание №3)}
Разработать конструкцию привода следящей системы по предложенной схеме
в соответствии с заданным вариантом.
\begin{table}[h]
    \begin{center}
        \begin{tabular}{|p{0.6\linewidth}|p{0.4\linewidth}|}
            \hline
            Вариант &   2 \\
            \hline
            Скорость вращения выходного вала \( \omega,\ c^{-1} \) & 5 \\
            \hline
            Ускорение вращения выходного вала \( \epsilon,\ c^{-2} \) & 36 \\
            % 0.5 & 360 & 4 & 110 & 0.022 & 0.04 \\
            \hline
            Момент инерции нагрузки \( J,\ \text{кгм}^2 \) & 0,015\\
            \hline
            Угол поворота выходного вала \( \phi \) & 120\\
            \hline
            Присоединительный диаметр \( d,\ \text{мм} \) & 50 \\
            \hline
            Тип потенциометра & ПТП или ППМЛ - выбирается самостоятельно
                с соответствующим обоснованием \\
            \hline
            Тип электродвигателя & Выбирать из серий ДИД, ДГ \\
            \hline
        \end{tabular}
        \caption{Исходные данные для расчёта}\label{tab:source}
    \end{center}
\end{table}

\newcommand{\Mnom}{M_{\text{ном}}}
\newcommand{\Mp}{M_{\text{п}}}
\newcommand{\nn}{n_{\text{ном}}}
\newcommand{\nv}{n_{\text{вых}}}

\setcounter{section}{1}
\section*{Расчётная часть}
\subsection{Предварительный расчёт электродвигателя}
    Сначала вычислим момент нагрузки на выходном валу:
    \[ M = J\epsilon = 0,015\cdot36 = 540\ \text{Нмм}\]
    Затем вычислим минимально возможную мощность двигателя, используя формулу
    \[ N = \xi\frac{M\omega}{\eta_p}. \]
    Приняв запас прочности \( \xi=1,1 \), а оценочный КПД редуктора \( \eta_p=0.8 \),
    получим \( N=3.71\ \text{Вт} \).

    Предварительно выберем двигатель ДГ-5ТА:
    \begin{table}[h!]
        \begin{center}
            \begin{tabular}{|p{0.6\linewidth}|p{0.4\linewidth}|}
                \hline
                Мощность \( N \), Вт & 5 \\
                \hline
                Номинальный момент \( \Mnom \), Нмм & 10 \\
                \hline
                Пусковой момент \( \Mp \), Нмм &    22 \\
                \hline
                Скорость вращения вала \( \nn \), об\( \backslash \)мин     & 6000 \\
                \hline
                Момент инерции ротора \( J_p \), \( \text{кгм}^2 \) & 4e-07 \\
                \hline
                Напряжение питания U, В & 36 \\
                \hline
            \end{tabular}
            \caption{Параметры двигателя}\label{tab:engine}
        \end{center}
    \end{table}

    \subsection{Кинематический и вспомогательный силовой расчёты}
    \subsubsection{Определение общего передаточного отношения}
        Вычислим общее передаточное отношение редуктора по формуле 
         \[ i_0 = \frac{\nn}{\nv}. \]
         Скорость вращения выходного вала находим по формуле 
          \[ \nv = \frac{30\omega}{\pi}. \]
         В результате получаем
          \[ i_0 = \frac{\nn\pi}{30\omega} = \frac{6000\cdot3,14}{30\cdot5} = 126.0\]
    
    \subsubsection{Определение числа ступеней}
        Число ступеней будем определять исходя из критериев минимизации
        момента инерции и габаритов, используя обьединённую формулу:
         \[ n = \frac{3+1,85}{2}\cdot\lg\ i0 \approx 5 \]
        Таким образом, редуктор будет иметь 5 ступеней.
    
    \subsubsection{Определение передаточных отношений каждой ступени}
        Распределение будет производить исходя из критерия минимизации
        момента инерции.
        Сперва вычислим среднее геометрические передаточное отношение:
        \newcommand{\iavr}{i_{\text{ср}}}
         \[ \iavr = \sqrt[n]{i_0} = \sqrt[5]{126.0} = 2.63 \]
        Затем рассчитаем непосредственно значения передаточного отношения для каждой ступени:
         \[ i_1 = \sqrt[4]{2\iavr} = 1.51 \]
        \[ i_2 = \sqrt{\iavr} = 1.62 \]
        \[ i_3 = \iavr = 2.63 \]
        \[ i_4 = \frac{\iavr^2}{i_2} = 4.27 \]
        \[ i_5 = \frac{\iavr^2}{i_1} = 4.57\]
    
    \subsubsection{Определение числа зубьев зубчатых колёс}
        Для подбора числа зубьев для шестерней имеет смысл брать минимальные из стандартного ряда,
        однако в процессе разработки конструкции делаются поправки исходя из необходимых
        минимальных расстояний между валами.
        Для колёс рассчитаем числа зубьев \( z_i \), воспользовавшись формулой
        \[ z_i = z_{i-1}i_j, \]
        где \( z_{i-1} \) - число зубьев соответствующей шестерни, а \( i_j \) - 
        передаточное отношение данной пары.
        
        По итогам расчётов получаем следущие значения:
        \begin{table}[h!]
            \begin{center}
                \begin{tabular}{p{0.13\linewidth}p{0.23\linewidth}p{0.2\linewidth}}
                    \hline
                    Номер ступени & Z для шестерни & Z для колеса \\
                    \hline
                    1   &   25 & 37 \\
                    % \hline
                    2   &   25 & 40 \\
                    % \hline
                    3   &   25 & 65 \\
                    % \hline
                    4   &   25 & 106 \\
                    % \hline
                    5   &   35 & 159 \\
                    \hline
                \end{tabular}
                \caption{Значения числа зубьев для зубчатых колёс}\label{tab:gears_z}
            \end{center}
        \end{table}
        
        В связи с тем, что стандартный ряд числа зубьев весьма дискретен, 
        результирующее передаточное отношение редуктора может отличаться от начального.
        Вычислим погрешность:
        \[ \Delta i = \frac{|i_0 - i_{\text{фактич.}}|}{i_0} =  
            \frac{|126.0-118.59|}{126.0} = 0.059\]
        
        Результат вполне удовлетворяет требования точности.
        
    \subsubsection{Расчёт крутящих моментов}
        Для расчёта дальнейших параметров зубчатых колёс необходимо найти крутящие моменты,
        действующие на каждом из валов редуктора. Для этого воспользуемся формулой
        \[ M_p = \frac{M_q}{i_{p-q}\eta_{pq}\eta_n}, \]
        где \( M_p, M_q \) - моменты нагрузки на p-м и q-м валах,
            \( i_{p-q} \) - передаточное отношение между валами,
            \( \eta_{pq},\ \eta_n \) - КПД передачи (для цилиндрической 0,98) и подшипников (0,99)\par
        
        По итогам вычислений получаем следующий результат (\( M_0 \) - вал двигателя):
        \begin{table}[h!]
            \begin{center}
                \begin{tabular}{p{0.2\linewidth}p{0.1\linewidth}p{0.1\linewidth}p{0.1\linewidth}p{0.1\linewidth}p{0.1\linewidth}p{0.1\linewidth}}
                    \hline
                    Номер вала, p & 0 & 1 & 2 & 3 & 4 & 5 \\
                    \hline
                    \( M_p \) & 5.0 & 7.33 & 11.52 & 29.4 & 121.79 & 540 \\
                    \hline
                \end{tabular}
                \caption{Оценочные значения моментов на валах}\label{tab:moments__shaft_estimate}
            \end{center}
        \end{table}

        Заметим, что \( M_0=5.0 \), что в 2 раза больше значения номинального момента двигателя,
        т.е. имеем большой запас по моменту двигателя.

    \subsubsection{Определение модуля зацепления}
        Модуль зацепления берётся по результатам расчётов зубьев на контактную и
        изгибную прочность. Расчёт на изгибную прочность проводится по наиболее нагруженной ступени в целях
        сокращения математических выкладок. Основная его формула
        \[ m \geq K_m\sqrt[3]{\frac{K\cdot M\cdot Y_F}{z\cdot\psi_{bm}\cdot[\sigma_F]}}, \]
        где m - ограничение снизу на искомый модуль,\par
            \( K_m \) - коэффициент, для прямозубых колес равный 1,4\par 
            \( K \) - коэффициент расчётной нагрузки (1,1..1,5)\par
            \( M \) - крутящий момент на соответствующем колесе\par
            \( Y_F \) - коэффициент формы зуба, выбирается по таблице/графику\par
            \( \psi_{bm} \) - коэффициент формы зубчатого венца, для мелкомодульных передач равен 3..16\par
            \( [\sigma_F] \) - допускаемое изгибное напряжение для материала\par
        
        Выберем материалы для колёс:
        \begin{table}[h!]
            \begin{center}
                \begin{tabular}{p{0.5\linewidth}p{0.2\linewidth}p{0.2\linewidth}}
                    \hline
                        & Шестерня  &   Колесо\\
                    \hline
                    Название    & сталь 40Х &   сталь 45 \\
                    Предел прочности \( \sigma_b \), МПа  & 1000 & 580 \\
                    Предел текучести \( \sigma_t \), МПа  &   830 & 360 \\
                    Предел выносливости \( \sigma_{-1} \), МПа  & 350.0 & 249.4 \\
                    *Изгибное напряжение \( [\sigma_f] \), МПа & 205.88 & 146.71 \\
                    *Контактное напряжение \( \sigma_H \), МПа & 602.0 & 428.97 \\
                    \hline
                    \emph{*будет вычислено далее}
                \end{tabular}
                \caption{Параметры материалов для зубчатых колёс}\label{tab:gear_materials}
            \end{center}
        \end{table}

        % \newpage
        Допускаемое изгибное напряжение \( \sigma_f \) вычисляется по формуле 
        \[ [\sigma_f] = \frac{\sigma_{-1}}{n}, \]
        где \( n \) - запас прочности. Принимаем \( n=1,7 \).\par
        
        Модуль для пары колёс вычисляется по тому из двух, которое обладает
        большей относительной характеристикой прочности.

       \begin{table}[h!]
            \begin{center}
                \begin{tabular}{p{0.5\linewidth}p{0.5\linewidth}}
                    Для шестерни:  &   Для колеса:\\
                    \[ \frac{Y_F}{[\sigma_f]} = 
                        \frac{3.73}{205.88} = 0.018\] &
                    \[ \frac{Y_F}{[\sigma_f]} = 
                    \frac{3.6}{146.71} = 0.025 \]\\                    
                \end{tabular}
            \end{center}
        \end{table}

        Считаем по колесу. Задаём \( K=1,3;\ \psi_{bm}=8 \):
        \[ m \geq 1,4\sqrt[3]{\frac{1,3\cdot 540\cdot 5.67}{159.95\cdot 8\cdot 146.71}}=0.38 \]
        
        Примем в качестве минимального момента ближайший табличный 0.4. Можно назначить его
        всем зубчатым колесам. Однако в процессе проектирования пришлось увеличить модуль некоторых
        пар зубчатых колес в целях увеличения их размера.

% TODO: расчёт на контактную прочность
    
    \subsubsection{Итоговый геометрический расчёт зубчатых передач}
        В этом пункте окончательно посчитаем оставшиеся параметры зубчатых передач,
        необходимые для построения чертежа, а именно: диаметры детительной окружности, окружностей вершин
        и впадин, ширину венца и межосевое расстояние.\par

        Диаметр делительной окружности рассчитывается по формуле
        \[ d = m\cdot z, \]
        где m и z - модуль и число зубьев соответствующего колеса.\par

        Диаметры окружностей вершин \( d_a \)и впадин \( d_f \)вычисляются по следующим формулам:
        \[ d_a = d + 2m,\qquad d_f = d - 2m(1+c),\]
        где с - коэффициент радиального зазора: 
            \( c=0,5 \) при \( m\leq 0,5 \), \( c=0,35 \) при \( 0,5<m<1 \).
        
        Для получения ширины колеса и межосевого расстояния служат формулы
        \[ b=\psi_{bm}m, \qquad  a = 0,5m(z_1+z_2)\]
        
        Приведём сводную таблицу с полными необходимыми характеристиками зубчатых колёс.
        Выберем материалы для колёс:
        \begin{table}[h!]
            \begin{center}
                \begin{tabular}{p{0.04\linewidth}|p{0.075\linewidth}p{0.075\linewidth}p{0.075\linewidth}p{0.075\linewidth}p{0.075\linewidth}p{0.075\linewidth}p{0.075\linewidth}p{0.075\linewidth}p{0.075\linewidth}p{0.075\linewidth}}
                    \hline
                    №   & 1&2&3&4&5&6&7&8&9&10\\
                    \hline
                    z & 25 & 37 & 25 & 40 & 25 & 106 & 25 & 106 & 35 & 159 \\
                    d & 10.0 & 10.0 & 10.0 & 16.2 & 12.5 & 32.88 & 12.5 & 53.37 & 21.0 & 95.97 \\
                    da & 10.8 & 15.9 & 10.8 & 17.0 & 13.5 & 33.88 & 13.5 & 54.37 & 22.2 & 95.97 \\
                    df & 8.8 & 8.8 & 8.8 & 15.0 & 11.0 & 31.38 & 11.0 & 51.87 & 19.38 & 94.35 \\
                    b & \multicolumn{2}{c}{3.2} &  \multicolumn{2}{c}{3.2} &  \multicolumn{2}{c}{4.0} & \multicolumn{2}{c}{4.0} &\multicolumn{2}{c}{4.8} \\
                    a & \multicolumn{2}{c}{12.55} & \multicolumn{2}{c}{13.1} & \multicolumn{2}{c}{22.69} & \multicolumn{2}{c}{32.94} & \multicolumn{2}{c}{58.48} \\
                    m & \multicolumn{2}{c}{0.4} & \multicolumn{2}{c}{0.4} & \multicolumn{2}{c}{0.5} & \multicolumn{2}{c}{0.5} & \multicolumn{2}{c}{0.6} \\
                    \hline
                \end{tabular}
                \caption{{Характеристики зубчатых колес}}\label{tab:gears_digest}
            \end{center}
        \end{table}

    
\subsection{Силовые расчёты}
    \subsubsection{Расчёт пружины люфтовыбирающего колеса}
        Люфтовыбирающим колесом решено сделать выходное, под номером 10.
        Расчёт пружины ведётся по необходимому рабочему усилию, которое
        вычисляется по формуле:
        \[ P_2 = \frac{\xi M_{\text{кр}}}{2(A\ cos\frac{180^\circ n}{z}
                - \frac{L}{2}\ sin\frac{180^\circ n}{z}},\]
        \begin{table}[h!]
            \begin{center}
                \begin{tabular}{p{0.025\linewidth}p{0.01\linewidth}p{0.8\linewidth}}
                    где & A & - расстояние от оси пружины до центра колеса,\\
                    & n & - число зубьев, на которое производится взаимное смещение составных частей колеса,\\
                    & L & - начальная длина пружины.
                \end{tabular}
            \end{center}
        \end{table}

        Приняв L=17мм, A=35мм, n=4, 
        z=159, \( \xi=2 \), получим:
        \[ P_2 = \frac{2\cdot540}{35\cdot cos\frac{180^\circ\cdot4}{159}
                - \frac{17}{2}\cdot sin\frac{180^\circ\cdot4}{159}} = 15.78\ H\]
        
        Сила пружины при максимальной деформации:
        \[ P_3 = \frac{P_2}{1-\delta} = \frac{15.78}{1-0.05} = 16.61\ H\]

        По найденному усилию по таблицам найдём подходящую пружину, склоняясь к более коротким
        и жёстким. Наш выбор: №173 II класса 2 разряда с параметрами: \( P_3 \) = 16.67 Н, d=0.45 мм, 
        D=75.033 мм, \( z_1 \)=75.033 Н/мм, \( f_3 \)=0.22 мм. Далее рассчитаем параметры
        пружины для нашего случая.

        Определим величину деформации \( F_2 \)при нагрузке \( P_2 \). Если 
        \[ L_1 =  L\ cos\frac{180^\circ n}{z} + 2A\ sin\frac{180^\circ n}{z} = 22.47\ \text{мм},\]
        то 
        \[ F_2 = L_1 - L = 5.47\ \text{мм}.\]

        Тогда жёсткость определяемой пружины будет
        \[ z = \frac{P_2}{F_2} = \frac{15.78}{5.47} = 2.88\ \text{Н/мм}; \]
        необходимое (и полное) число рабочих витков
        \[ n_1 = n = \frac{z_1}{z} = 26;\]
        длина пружины в свободном состоянии
        \[ H_0 = (n_1 + 1)d = (26 + 1)\cdot 0.45= 12.15\ \text{мм}; \]
        полная длина с крючками
        \[ L' = H_0 + 2D = 12.15 + 2\cdot 2.2 = 16.55\ \text{мм}. \]
        Следовательно, конструктивные размеры выбраны верно.
        

        
        
        
        
        
        
        

    
\end{document}